\documentclass[11pt]{article}  
\usepackage{a4wide,ngerman,url,amsmath,bbm,graphicx,enumitem}
\usepackage[utf8]{inputenc}

\newcommand{\br}[1]{\ensuremath{\left(#1\right)}}
\newcommand{\cbr}[1]{\ensuremath{\left\{#1\right\}}}
\newcommand{\ii}{\mathrm{i}}
\newcommand{\N}{\mathbbm{N}}
\newcommand{\Z}{\mathbbm{Z}}
\newcommand{\Q}{\mathbbm{Q}}
\newcommand{\C}{\mathbbm{C}}
\def\pw{{\char94}}

\newenvironment{code}{\tt \begin{tabbing}
\hskip12pt\=\hskip12pt\=\hskip12pt\=\hskip12pt\=\hskip5cm\=\hskip5cm\=\kill}
{\end{tabbing}}

\parindent0pt
\parskip3pt

\author{Hans-Gert Gräbe}
\title{Zur Konstruktion regulärer Polygone} 
\date{Version vom 27.12.2020}

\begin{document} 
\maketitle         

\section{Zahlentheoretische Vorbereitungen}

Für eine natürliche Zahl $n>1$ bilden die primen Restklassen
\begin{gather*}
  \Z_n^\ast=\cbr{a \pmod n \mid 0<a<n,\ \gcd(a,n)=1}
\end{gather*}
bekanntlich eine multiplikative Gruppe, die für eine Primzahl $p$ darüber
hinaus zyklisch ist, d.h. es gibt Elemente $g\in \Z_p^\ast$ mit
\begin{gather*}
  \Z_p^\ast=\cbr{1,g,g^2,\dots,g^{p-2}}.
\end{gather*}
Solche Elemente werden als \emph{primitive Wurzeln} modulo $p$ bezeichnet;
ihre Anzahl ist gleich der Anzahl der primen Restklassen modulo $p-1$, denn
man überzeugt sich leicht, dass mit $g$ auch jedes Element $g'\equiv g^a
\pmod{p-1}$ mit $\gcd(a,p-1)=1$ ebenfalls die Gruppe $\Z_p^\ast$ erzeugt.

Die Existenz solcher $g$ charakterisiert zugleich Primzahlen, da die Ordnung
$\left|\Z_n^\ast\right|$ der Gruppe der primen Restklassen für
zusammengesetzte Zahlen stets kleiner als $n-1$ ist.

Dies ist auch die Basis für Primzahlzertifikate, da die einschlägigen
Primtestalgorithmen zusammengesetzte Zahlen $n$ sicher erkennen und einen
\emph{Zeugen} $W(n)$ zu produzieren vermögen, mit dem die Zusammengesetztheit
schnell nachgewiesen werden kann, für Primzahlen $p$ aber nur die Antwort
liefern „mit sehr hoher Wahrscheinlichkeit prim“. Hier liefert eine Restklasse
$g \pmod p$ samt dem Beweis, dass $\mathrm{ord}(g)=p-1$ in $\Z_p^\ast$ gilt,
dann den Beweis, dass $p$ eine Primzahl ist.

$\mathrm{ord}(g)=p-1$ gilt genau dann, wenn $g^d\not\equiv 1\pmod p$ für alle
$d=(p-1)/q$ gilt, wobei $q$ die Primteiler von $p-1$ durchläuft. Auch dies
kann schnell nachgeprüft werden, wenn die Primteiler von $p-1$ bekannt sind.
Diese lassen sich erstaunlich oft für den Vorgänger $p-1$ einer Primzahl $p$
bestimmen. Besonders einfach ist dies natürlich für Primzahlen der Gestalt
$p=2^k+1$, denn dann hat $p-1$ den einzigen Primteiler $2$, und es kann stets
$g=3$ als primitive Wurzel genommen werden. 

Für Details sei auf \cite{Graebe2012} verwiesen. 

$p=2^k+1$ ist höchstens dann eine Primzahl, wenn $k$ selbst eine Zweierpotenz
$k=2^m$ ist.  Solche Primzahlen werden als \textsc{Fermat}sche Primzahlen
$F_m=2^{2^m}+1$ bezeichnet. Es ist bekannt, dass diese Zahlen für
$m\in\cbr{0,1,2,3,4}$ prim sind. Für $5\le m\le 32$ (Stand Oktober 2020) ist
bekannt, dass sie zusammengesetzt sind, für $m=33$ (Stand Oktober 2020) ist
diese Frage nicht entschieden.  Man geht allerdings davon aus, dass alle
weiteren \textsc{Fermat}-Zahlen zusammengesetzt sind.

Für die Konstruierbarkeit regulärer Polygone mit Zirkel und Lineal ist diese
Frage allerdings immens wichtig, denn \textsc{Gauß} hat gezeigt, dass für
primes $p$ genau die regulären $p$-Ecke mit Zirkel und Lineal konstruierbar
sind, für die $p$ eine \textsc{Fermat}sche Primzahl ist.

Ein reguläres $p$-Eck kann aus einer komplexen $p$-ten Einheitswurzel
$\zeta_p\in\C$ erzeugt werden, denn der zugehörige Punkt $Z_p$ in der
komplexen Ebene liegt auf dem Einheitskreis um den Ursprung, und durch
wiederholtes Abtragen des Abstands $EZ_p$ kann schließlich das $p$-Eck
konstruiert werden, wobei $E=1$ für den Einheitspunkt auf der reellen Achse
steht.

$\zeta_p\in\C$ ist dabei eine algebraische Zahl, denn $\zeta_p$ ist Nullstelle
des Polynoms $z^p-1=(z-1)\cdot \phi_p(z)$ und wegen $\zeta_p\neq 1$ auch
Nullstelle des über $\Z$ irreduziblen Polynoms
\begin{gather*}
  \phi_p(z)=z^{p-1}+z^{p-2}+\dots+z^2+z+1,
\end{gather*}
also (über $\Q$) eine algebraische Zahl vom Grad $p-1$.  Für die
\textsc{Fermat}sche Primzahl $p=F_m$ gilt gerade $p-1=2^k=2^{2^m}$ und die
Idee der Verwirklichung der Konstruktion eines regulären $p$-Ecks besteht
darin, eine Folge algebraischer Zahlen $A_1, A_2,\ldots$ zu finden, für die
$K_1=\Q(A_1)$, $K_2=K_1(A_2), \ldots, K_k=K_{k-1}(A_k)=\Q(\zeta_p)$ jeweils
quadratische Erweiterungen sind. Komplexe Zahlen aus solchen quadratischen
Erweiterungen lassen sich in der komplexen Zahlenebene aus bis dahin
konstruierten Zahlen als Schnittpunkte geeigneter Kreise erzeugen.

Die Nullstellen $N_p=\cbr{\zeta_p^i, 1\le i\le p-1}$ des irreduziblen Polynoms
$\phi_p(z)$ lassen sich als $N_p=\cbr{\zeta_p^i, i\in\Z_p^\ast}$ und damit als
\begin{gather*}
  N_p=\cbr{\zeta_p,\zeta_p^g, \zeta_p^{g^2}, \dots, \zeta_p^{g^{p-2}}}
\end{gather*}
darstellen, wobei $g$ eine primitive Wurzel modulo $p$ ist und die Exponenten
$g^t$ wegen $\zeta_p^p=1$ natürlich modulo $p$ reduziert werden können.  Für
$p=F_m$ und $g=3$ konstruieren wir die sukzessiven Erzeugenden quadratischer
Erweiterungen $A_1, A_2$ usw. als geeignete Summen von Elementen aus $N_p$.

Dies führen wir in den nächsten Kapiteln für $m\in\cbr{1,2,3,4}$ genauer aus.
Für die dabei erforderlichen Rechnungen in Erweiterungskörpern von $\Q$ setzen
wir das freie Computeralgebrasystem \textsc{Maxima} ein, für Details des 
algorithmischen und programmiertechnischen Hintergrunds wird auf das Skript
\cite{Graebe2020} verwiesen, dort insbesondere auf das Kapitel 4.

\section{Reguläres 5-Eck}

Wir setzen $A_1=\zeta_5+\zeta_5^4$, wobei sich der zweite Exponent aus
$3^2\equiv 4 \pmod{5}$ ergeben hat. Wegen
$\zeta_5^4+\zeta_5^3+\zeta_5^2+\zeta_5+1=0$ ist dann
$\zeta_5^3+\zeta_5^2=-(A_1+1)$ und
\begin{gather*}
  -A_1(A_1+1)=\br{\zeta_5+\zeta_5^4}\br{\zeta_5^3+\zeta_5^2}
  =\zeta_5^4+\zeta_5^3+\zeta_5^7+\zeta_5^6 =
  \zeta_5^4+\zeta_5^3+\zeta_5^2+\zeta_5 = -1.
\end{gather*}
$A_1$ erfüllt also die quadratische Gleichung $A_1(A_1+1)-1=0$ und ist somit
eine algebraische Zahl vom Grad 2 über $\Q$.  Ähnlich erfüllt $\zeta_5$ die
Beziehung
\begin{gather*}
  1=\zeta_5^5 = \zeta_5\cdot \zeta_5^4 = \zeta_5(A_1-\zeta_5)
\end{gather*}
und liegt damit in einer quadratischen Erweiterung von $K_1=\Q(A_1)$.

Hier dieselben Rechnungen mit \textsc{Maxima}.
\begin{code}
tellrat((z5\pw5-1)/(z5-1));\\
algebraic:true;\\
A1:sum(z5\pw(power\_mod(3,(2*i),5)),i,0,1);
\end{code}
\begin{gather*}
  z+z^4
\end{gather*}
\begin{code}
ratsimp(A1*(A1+1));  /* 1 */\\
ratsimp(z5*(A1-z5)); /* 1 */
\end{code}
$A_1$ ist also Nullstelle des Polynoms $f_1=z(z+1)-1\in\Q[z]$ und $\zeta_5$
Nullstelle des Polynoms $f_2=z(A_1-z)-1\in K_1[z]$. 
\begin{code}
f1:B1*(B1+1)-1;   \\
f2:B2*(B1-B2)-1;  \\
s1:solve(f1,B1);
\end{code}
\begin{gather*}
  B_1 = \frac12\br{\sqrt5-1}
\end{gather*}
\begin{code}
s2:solve(f2,B2);
\end{code}
\begin{gather*}
  \zeta_5 = B_2 = \frac12\br{\sqrt{B_1^2-4}+B_1}
\end{gather*}
\begin{code}
expand(subst(s1[2],s2[2]));
\end{code}
\begin{gather*}
  \zeta_5 = \frac14\br{\sqrt5-1}+\frac12\sqrt{\frac12\br{-5-\sqrt5}}
  = \frac{\sqrt5-1}{4}+\ii\sqrt{\frac{5+\sqrt5}{8}} = \cos\br{\frac{2\pi}{5}}
  + \ii\sin\br{\frac{2\pi}{5}}
\end{gather*}
Die Konstruktion ist in diesem Fall einfach --
$\cos\br{\frac{2\pi}{5}}=\frac{\sqrt5-1}{4}$ ist eine algebraische Zahl vom
Grad 2 und lässt sich einfach aus einem rechtwinkligen $(2\times 1)$-Dreieck
konstruieren, dessen Hypotenuse gerade die Länge $\sqrt{1^2+2^2}=\sqrt{5}$
hat.  Auch die Konstruktion der Länge $\sin\br{\frac{2\pi}{5}}
=\sqrt{\frac{5+\sqrt5}{8}}$ ist nicht schwierig -- wir konstruieren zunächst
wie eben eine Strecke $s$ der Länge $\frac{5+\sqrt5}{8}$ und verwandeln danach
ein $(s\times 1)$-Rechteck in ein flächengleiches Quadrat, das dann genau die
Seitenlänge $\sin\br{\frac{2\pi}{5}}$ hat.  Auf ähnliche Weise lassen sich
alle Streckenlängen konstruieren, die sich als algebraische Zahlen vom Grad 2
aus gegebenen Streckenlängen darstellen lassen. 

\section{Reguläres 17-Eck}

Mit $\zeta=\zeta_{17}$ konstruieren wir die drei Größen nach demselben Muster
wie oben
\begin{align*}
  A_1=\sum_{i=0}^7{\zeta^{3^{2i}}},\
  A_2=\sum_{i=0}^3{\zeta^{3^{4i}}},\
  A_3=\sum_{i=0}^1{\zeta^{3^{8i}}}=\zeta+\zeta^{16}=2\cos\br{\frac{2\pi}{17}}
\end{align*}

\begin{code}  
tellrat((z1\pw(17)-1)/(z1-1));\\
algebraic:true;\\
A1:sum(z1\pw(power\_mod(3,(2*i),17)),i,0,7);\\
A2:sum(z1\pw(power\_mod(3,(4*i),17)),i,0,3);\\
A3:sum(z1\pw(power\_mod(3,(8*i),17)),i,0,1);
\end{code}
Wir finden wieder einfach die quadratischen Gleichungen für $A_1$ über $\Q$
und für $A_2$ über $K_1=\Q[A_1]$ 
\begin{code}  
ratsimp(A1*(A1+1));  /* 4 */\\
ratsimp(A2*(A2-A1)); /* 1 */
\end{code}
Etwas mehr Mühe bereitet es zu zeigen, dass $A_3(A_3-A_2)$ in
$K_2=K_1[A_2]=\Q[A_1,A_2]$ liegt. $K_2$ ist eine Körpererweiterung vom Grad 4
über $Q$ mit der reduzierten Basis $T_\text{red}=\cbr{1,A_1,A_2,A_1A_2}$, aus
der wir durch einen Ansatz mit unbestimmten Koeffizienten die erforderliche
$\Q$-lineare Kombination für $A_3(A_3-A_2)$ bestimmen.  $\zeta$ liegt
schließlich in einer quadratischen Erweiterung von $K_3=K_2[A_3]$.
\begin{code}  
ratsimp(2*A3*(A3-A2)-3-A1+A2+A1*A2); /* 0 */\\
ratsimp(z1*(z1-A3)); /* -1 */
\end{code}
Wir können daraus die Ausdrücke für $A_1$, $A_2$, $A_3$ und $\zeta$ als
geschachtelte Quadratwurzeln leicht bestimmen.
\begin{code}  
f1:B1*(B1+1)-4;   \\
f2:B2*(B2-B1)-1;  \\
f3:2*B3*(B2-B3)+3+B1-(B2+B1*B2);\\  
f4:B4*(B4-B3)+1;  \\
s1:solve(f1,B1);  /* B1 = (sqrt(17)-1)/2 */\\
s2:solve(f2,B2);  /* B2 = (sqrt(B1\pw2+4)+B1)/2 */\\
s3:solve(f3,B3);  /* B3 = (sqrt(B2\pw2+((-2*B1)-2)*B2+2*B1+6)+B2)/2 */\\
s4:solve(f4,B4);  /* B4 = (sqrt(B3\pw2-4)+B3)/2 */\\[1em]
u1:ratsimp(expand(subst(s3[2],s4[2])));
u2:ratsimp(expand(subst(s2[2],u1)));
u3:ratsimp(expand(subst(s1[2],u2)));
\end{code}
Mit Blick auf die schiere Größe dieser Ausdrücke sind sie hier nicht reproduziert


\section{Reguläres 257-Eck}

Die moderne Begründung der Konstruierbarkeit des regulären 257-Ecks mit Zirkel
und Lineal erfolgt durch Folgerungen aus der \textsc{Galois}-Theorie.

Der Mittelpunkt des 257-Ecks liegt im Ursprung eines komplexen
Koordinatensystems, d.h. die reellen Zahlen finden sich auf der Abszissenachse
und die Vielfachen von $\ii$ mit $\ii^{2}=-1$ auf der Ordinatenachse.

Alle Eckpunkte des 257-Ecks liegen auf einen Kreis. Zur Vereinfachung wählt
man den Einheitskreis mit dem Radius $r = 1$. Das 257-Eck wird so gedreht,
dass ein Eckpunkt der Punkt $(0,1)$ ist. Die Eckpunkte des 257-Ecks erfüllen
die komplexe Gleichung $z^{257} = 1$.

Die Gleichung $z^{257} =1$ soll im Bereich der komplexen Zahlen gelöst werden,
also $z^{257} - 1 = 0$. Der Winkel $\alpha$ mit dem Scheitelpunkt im Ursprung
des Koordinatensystems zwischen Abszissenachse und Strahl durch den Punkt
$P_1$, beträgt in Bogenmaß $\alpha= \frac{2\pi}{257}$. Also ist $z =
\cos(\alpha) + \ii\sin(\alpha)$.

So ist nach der \textsc{Euler}schen Identität $z =
\mathrm{e}^{\frac{2\pi\ii}{257}}$.  Es ist überhaupt nicht schwierig, wenn man
es mit komplexen Zahlen zu tun hat. Man kann über den Realteil von $z$, also
$\cos(\alpha)$, der auf der Abszissenachse zu finden ist, den Schnittpunkt mit
dem Einheitskreis finden. Dieser Punkt ist dann einer der Eckpunkte des
regulären 257-Ecks.

Im Bereich der komplexen Zahlen hat die Gleichung $z^{257} = 1$ nach dem
\textsc{Gauss}schen Fundamentalsatz der Algebra 257 Lösungen. Wie man sofort
sieht, ist $z = 1$ eine (triviale) reelle Lösung. Die anderen 256 Lösungen
ergeben sich als Nullstellen des Polynoms
\begin{gather*}
   \frac{z^{257}-1}{z-1}=z^{256}+z^{255}+\ldots+z^3+z^2+z+1.\tag{2}
\end{gather*}
Diese Nullstellen $z_i$, $i=0,\ldots,256$, lassen sich alle als $z_i=\zeta^i$
für eine geeignete komplexe Zahl $\zeta$ darstellen; die Multiplikation zweier
Nullstellen $z_i\cdot z_j=\zeta^{i+j}$ entspricht dabei der Addition der
Exponenten, wobei wegen $\zeta^{257}=1$ der Exponent $i+j$ modulo 257
reduziert, also in der additiven Gruppe des Restklassenrings $\Z_{257}$
gerechnet werden kann.

Kriterium der Auswahl ist der Index der $z_i$ ($i = 1, 2, ..., 257$). Man
verwendet die Restklassen der Indizes modulo 2, also 0 oder 1. (Die Restklasse
kommt in den Indizes der Mengen $M_0$ und $M_1$ zum Ausdruck). So nehmen wir
alle 128 Lösungen $z_i = z^{i \pmod{257}}$ mit geraden Indizes in die Menge
$M_0$, die restlichen 128 Lösungen in die andere Menge $M_1$. Es sei 
\begin{align*}
  A_0&= \sum_{0<i<256, i\ \text{ungerade}}{z^{3^i \pmod{257}}}
  =z^3+z^{27}+\ldots+z^{238}+z^{86} \quad \text{und}\\
  A_1 &= \sum_{0\le i <256, i\ \text{gerade}}{z^{3^i \pmod{257}}} =
  z^1+z^9+\ldots+z^{165}+z^{200}.
\end{align*}
% Maxima:
%% tellrat((z^(257)-1)/(z-1));
%% algebraic:true;
%% A0:sum(z^(power_mod(3,(2*i+1),257)),i,0,127);
%% A1:sum(z^(power_mod(3,(2*i),257)),i,0,127);
%% ratsimp(A0+A1);  /*  - 1 */
%% ratsimp(A0*A1);  /* - 64 */

Wie man sofort sieht, ist $A_0 + A_1 = -1$. Das Produkt $A_0 \cdot A_1$ ergibt
nach Ausmultiplizieren eine Summe von $128 \cdot 128 = 16384$ Summanden, die
sich zu $-64$ vereinfachen lässt. Beides kann mit \textsc{Maxima}\footnote{Ein
  freies CAS, das von Lehrern gern eingesetzt wird,
  \url{http://maxima.sourceforge.net/}} schnell nachgeprüft werden:
\begin{code}
tellrat((z\pw(257)-1)/(z-1));\\
algebraic:true;\\
A0:sum(z\pw(power\_mod(3,(2*i+1),257)),i,0,127);\\
A1:sum(z\pw(power\_mod(3,(2*i),257)),i,0,127);\\
ratsimp(A0+A1); /*  -1 */\\
ratsimp(A0*A1); /* -64 */
\end{code}

Die Gleichungen $A_0 + A_1 = -1$ und $A_0 \cdot A_1 = -64$ führen nach dem
\textsc{Vieta}schen Wurzelsatz zu der quadratischen Gleichung $s^2 + s - 64 =
0$ mit 
\begin{gather*}
  A_0=\frac{-1+\sqrt{257}}{2}\quad\text{und}\quad A_1=\frac{-1-\sqrt{257}}{2}.
\end{gather*}
$A_0$ und $A_1$ sind beides reelle Zahlen. Man findet sie auf der
Abzissenachse des komplexen Koordinatensystems (siehe Schritt 36 der
Konstruktion des regulären 257-Ecks).

Als nächstes zerlegt man die Menge der Lösungen mit geraden Indices $M_{0}$
weiter in zwei gleichmächtige Mengen $M_{0,0}$ und $M_{0,2}$. In der Menge
$M_{0,0}$ sind jetzt alle Lösungen mit durch 4 teilbaren Indizes und in der
Menge $M_{0,2}$ alle Lösungen mit geradem, aber nicht durch 4 teilbaren
Indizes.  Man wählt den Ansatz 
\begin{align*}
  B_0 &= z^1 + z^{81} + z^{136} + ... + z^{240} + z^{165}\quad\text{und}\\ B_2
  &= z^9 + z^{215} + z^{196} + ... + z^{104} + z^{200}.
\end{align*}
Es ist $B_0 + B_2 = A_0$ und $B_0 \cdot B_2 = -164$. In den Tabellen 3 und 4
ist diese Multiplikation vollständig dargestellt.

\textsc{Maxima} liefert allerdings ein anderes Ergebnis (Fortsetzung der
Rechnung oben): 
\begin{code}
B0:sum(z\pw(power\_mod(3,(4*i),257)),i,0,63);\\
B2:sum(z\pw(power\_mod(3,(4*i+2),257)),i,0,63);\\
ratsimp(B0+B2+A0); /*  - 1 */\\
ratsimp(B0*B2);  /* - 16 */
\end{code}

% Maxima: Fortsetzung
%% B0:sum(z^(power_mod(3,(4*i),257)),i,0,63);
%% B2:sum(z^(power_mod(3,(4*i+2),257)),i,0,63);
%% ratsimp(B0+B2+A0); /*  - 1 */
%% ratsimp(B0*B2);  /* - 16 */

Daraus erhält man die quadratische Gleichung $s^{2}- A_{0} \cdot s - 16 = 0$
mit den Lösungen
\begin{gather*}
  B_0 = \frac{A_0}{2}+ \sqrt{\br{\frac{A_0}{2}}^2+16}\quad\text{und}\quad B_2
  = \frac{A_0}{2}- \sqrt{\br{\frac{A_0}{2}}^2+16}.
\end{gather*}

\begin{thebibliography}{xxx}
\bibitem{Graebe2012} Hans-Gert Gräbe. Algorithmen für Zahlen und Primzahlen.
  Eagle Verlag, Leipzig 2012. ISBN 978-3-937219-58-5. 
\bibitem{Graebe2020} Hans-Gert Gräbe.  Skript zum Kurs \emph{Einführung in das
  symbolische Rechnen}, Sommersemester 2020.
  \url{http://www.informatik.uni-leipzig.de/~graebe/skripte/esr20.pdf}
\end{thebibliography}
\end{document}
