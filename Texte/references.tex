
\begin{thebibliography}{xxx}
\bibitem{GaussTagebuch} Mathematisches Tagebuch 1796-1814 von Carl Friedrich
  Gauss. Mit einer historischen Einführung von Kurt-R. Biermann. Ostwalds
  Klassiker der exakten Wissenschaften, Band 256. Leipzig 1976.
\bibitem{Patzschke2002} Patzschke, Jürgen: Carl Friedrich Gauss und das
  Siebzehneck -- Teil I. In \emph{Wurzel}, Heft 5, 2002, S. 15-24.
\bibitem{Paucker1822} Paucker, Georg: Geometrische Verzeichnung des
  regelmässigen Siebzehn-Ecks und Zweyhundertsiebenundfunfzig-Ecks in den
  Kreis. In \emph{Jahresverhandlungen der kurländischen Gesellschaft für
    Literatur und Kunst}, Zweyter Band, Mitau 1822, S. 160-219.
\bibitem{Richelot1832} Richelot, Friedrich Julius: De resolutione algebraica
  aequationis $x^{257} = 1$, sive de divisione circuli per bisectionem anguli
  septies repetitam in partes 257 inter se aequales commentatio coronata. In
  \emph{Journal für die reine und angewandte Mathematik}, Nr. 9, 1832,
  S. 1-26, 146-161, 209-230 und 337-358.
\bibitem{deTemple1991} DeTemple, Duane W.: Carlyle Circles and the Lemoine
  Simplicity of Polygonal Constructions. In \emph{Amer. Math.  Monthly},
  vol. 98, 1991, S. 97-108.
\bibitem{Gottlieb1999} Gottlieb, Christian: The Simple and Straightforward
  Construction of the Regular 257-gon. In \emph{Mathematical Intelligencer},
  vol. 21, No. 1, 1999, S. 31-37.
\bibitem{Bishop1978} Bishop, Wayne: How to construct a regular polygon. In
  \emph{Amer. Math. Monthly}, vol. 85, 1978, S. 186-188.
\bibitem{Polster2016} Polster, Steffen: Software \emph{Mathematik Alpha 2016},
  \url{http://mathematikalpha.de}.
\bibitem{Hermes1898} [17] Hermes, Johann Gustav: Diarium zur Kreisteilung,
  Königsberg 1879, (1879 begonnen - d. A.)
\bibitem{Hermes1894} [18] Hermes, Johann Gustav: Ueber die Teilung des Kreises
  in 65537 gleiche Teile. In \emph{Nachrichten von der Gesellschaft der
  Wissenschaften zu Göttingen}, Mathematisch-Physikalische Klasse. Göttingen,
  1894, S. 170–186.
%[19] http://facstaff.susqu.edu/brakke/constructions/65537-gon.m.txt (abgerufen am 06.06.2017)
\end{thebibliography}
