\documentclass[11pt]{article}  
\usepackage{a4wide,ngerman,url,amsmath,bbm,graphicx}
\usepackage[utf8]{inputenc}

\newcommand{\br}[1]{\ensuremath{\left(#1\right)}}
\newcommand{\N}{\mathbbm{N}}
\newcommand{\ggT}{\mathrm{ggT}}
\newcommand{\HGG}[1]{\begin{quote} Einfügen: #1 \end{quote}}

\parindent0pt
\parskip3pt

\author{Bernd Winter}
\title{Zur Konstruktion regulärer Polygone, insbesondere des regulären
  17-Ecks, 257-Ecks und 65537-Ecks} 
\date{Erstellt 2012, letztes Update 12.06.2019}

\begin{document} 
\maketitle         
\begin{quote}
  Quelle: \url{https://www.mathematik-olympiaden.de/moev/moev_material/}\\
  \url{Konstruktion17/index.html}

  Dort auch ein Video zur Konstruktion des regulären 257-Ecks, Impressionen
  zum legendären Koffer mit der Arbeit zum 65537-Eck von HERMES in Göttingen. 
\end{quote}

Reguläre Polygone oder $n$-Ecke sind solche $n$-Ecke mit gleicher Seitenlänge
$a_n$.  Die klassischen Konstruktionswerkzeuge nach \textsc{Euklid} (etwa 365
v. Chr. bis etwa 300 v. Chr.) sind Zirkel und Lineal (ohne Messfunktion).
Damit können Kreise um einen (Mittel)punkt durch einen zweiten Punkt, Geraden,
Strahlen und Strecken durch zwei Punkte, Schnittpunkte zweier Kreise, zweier
Geraden oder zwischen Kreis und Gerade (Strahl oder Strecke) konstruiert
werden.

Wenn man ein $x$-$y$-Koordinatensystem in der Zeichenebene einführt, kann man
z. B. Kreise mittels der Gleichung $\br{x-x_m}^2+\br{y-y_m}^2=r^2$ ($x_m$,
$y_m$ Koordinaten des Kreismittelpunkts, $r$ Kreisradius) oder Geraden mittels
der Gleichung $ax + by = c$, ($a \neq 0$ oder $b \neq 0$) darstellen.
Schnittpunkte dieser Objekte lassen sich mit Gleichungssystemen aus derartigen
Gleichungen oder Folgen derartigen Gleichungssysteme beschreiben. (Es kommen
also z. B. keine Gleichungen dritten Grades vor.) Die Lösungen derartiger
Systeme, sofern diese existieren, sind also endlich lange Terme, die reelle
Zahlen, evtl. verknüpft durch die vier Grundrechenarten und Quadratwurzeln
sowie Verschachtelungen davon enthalten. (Siehe z. B. den Ausdruck (1)).

Größtenteils bereits seit dem Altertum (EUKLID) sind Konstruktionen mit Zirkel
und Lineal u. a. für einige reguläre $n$-Ecke bekannt, insbesondere für das
reguläre 3-, 4- und 5-Eck: Man erkennt sofort die „Verwandtschaft“ der
regulären $n$-Ecke in den „Familien“ der regulären 3-, 6-, 12-, 24-, ... -Ecke
oder der 4-, 8-, 16-, ... -Ecke oder der 5-, 10-, 20-, ... -Ecke.

Durch geschickte Kombination ließen sich z. B. auch reguläre 15-Ecke
konstruieren: Nach dem Lemma von BÉZOUT (1730--1783) gibt es für zwei
natürliche Zahlen $a$ und $b$, hier 3 und 5, zwei ganze Zahlen $s$ und $t$, so
dass sich der größte gemeinsame Teiler $\mathrm{ggT}(a,b)$ der Zahlen $a$ und
$b$ als $\ggT(a;b) = s \cdot a + t \cdot b$ darstellen lässt. Da Primzahlen
und somit auch FERMATsche Primzahlen stets teilerfremd zueinander sind, ist
$\ggT(3; 5) = 1$. So ist $1 = s \cdot 3 + t \cdot 5$. Man kann die Zahlen $s$
und $t$ mit dem erweiterten EUKLIDischen Algorithmus bestimmen:
\begin{align*}  
  5 : 3 &= 1\ \text{Rest}\ 2 & 	2 &= 5 - 1 \cdot 3\\{}
  3 : 2 &= 1\ \text{Rest}\ 1 &	1 &= 3 - 1 \cdot 2\\{}
  && 1 &= 3 - 1 \cdot (5 - 1 \cdot 3)\\{}
  && 1 &= 2 \cdot 3 - 1 \cdot 5
\end{align*}
So ist $s = +2$ und $t = -1$.

Anschaulich ausgedrückt heißt das: Man zeichne den (Bestimmungs)winkel
($72^\circ$) des regulären 5-Ecks im mathematisch positiven Drehsinn (gegen
den Uhrzeigersinn) und füge einen weiteren Winkel mit dieser Größe an. Man
füge in entgegengesetztem Drehsinn den (Bestimmungs)winkel ($120^\circ$) des
regulären 3-Ecks an. Der Gesamtwinkel ist der Bestimmungswinkel des regulären
15-Ecks.

\HGG{Animiertes gif images/15-Eck.gif hier einbinden}

Sofort fällt auf, dass in diesen Mengen u. a. das reguläre 7-, 9- und 17-Eck
nicht enthalten sind. Bis zum Ende des 18. Jahrhunderts war die Frage der
Konstruierbarkeit dieser regulären $n$-Ecke ungelöst.

GAUSS (1777-1855) konnte dieses Problem lösen. Er schrieb 1819 dazu: „Die
Geschichte jener Entdeckung ist bisher nirgends von mir öffentlich erwähnt,
ich kann es aber sehr genau angeben. Der Tag war der 29. März 1796, und der
Zufall hatte gar keinen Anteil daran. Schon früher war alles ... von mir
gefunden ..., und zwar im Winter 1796 (meinem ersten Semester in Göttingen),
ohne daß ich den Tag aufgezeichnet hätte. Durch angestrengtes Nachdenken über
den Zusammenhang aller Wurzeln (heutiger Begriff: Lösungen -- d. A.)
untereinander nach arithmetischen Gründen glückte es mir, bei einem
Ferienaufenthalt in Braunschweig am Morgen des gedachten Tages (ehe ich aus
dem Bette aufgestanden war) diesen Zusammenhang auf das klarste anzuschauen,
so daß ich die spezielle Anwendung auf das 17-Eck und die numerische
Bestätigung auf der Stelle machen konnte.“ ([1], sinnwahrend vom Autor
gekürzt.)

Der im Text beschriebene Ausdruck ist: 
\begin{align*}
  \cos\br{\frac{2\pi}{17}}&= \frac{1}{16}
  \Bigg(-1+\sqrt{17}+\sqrt{{2\br{17-\sqrt{17}}}}+\\ &\qquad
  2\sqrt{17+3\sqrt{17}-\sqrt{2\br{17-\sqrt{17}}}-2\sqrt{2\br{17+\sqrt{17}}}}
  \Bigg) \tag{1}.
\end{align*}

Es sei auf die ausführliche lesenswerte Herleitung der Gleichung durch
PATSCHKE [2] verwiesen. Der Winkel $\frac{2\pi}{17}$ ist der in Bogenmaß
ausgedrückte Bestimmungswinkel des regulären 17-Ecks.

Ausführlich hat GAUSS seine Theorie in seinem Werk „Disquisitiones
arithmeticae“ 1801 dargestellt: Ausgangspunkt ist die Vorstellung, dass die
Konstruktion des regulären $n$-Ecks im Einheitskreis äquivalent zur Bestimmung
der Lösungen der Kreisteilunggleichung $x^n - 1 = 0$ ($n\in\N$) ist.

Er fand dann die hinreichende Bedingung für die Konstruktion regelmäßiger
Polygone. Er vermutete, dass die Bedingung auch notwendig ist, gab allerdings
keinen Beweis an. WANTZEL (1814-1848) holte dies 1837 [3] nach.

Diese notwendige und hinreichende Bedingung für die Konstruierbarkeit des
regulären $n$-Ecks an die Zahl $n$ lautet:
\begin{quote}
  $n$ ist eine natürliche Zahl größer als 2 und $n$ ist ein Produkt aus einer
  Potenz von 2 mit einer nichtnegativen ganzen Zahl als Exponenten und
  möglicherweise als weitere Faktoren voneinander verschiedene FERMATsche
  Primzahlen.
\end{quote}
Diese speziellen Primzahlen sind nach dem französischen Mathematiker und
Juristen FERMAT (1601--1665) benannt.

Potenzen von 2 sind 4, 8, 16, ...

Man erkennt auch sofort die oben genannten „Familien“ konstruierbarer
regulärer $n$-Ecke wieder.

FERMATsche Primzahlen $n =2^{\br{2^k}}, k\in\N$.
\begin{align*}
  k = 0 &\rightarrow{} n = 3\\
  k = 1 &\rightarrow{} n = 5\\
  k = 2 &\rightarrow{} n = 17\\
  k = 3 &\rightarrow{} n = 257\\
  k = 4 &\rightarrow{} n = 65537
\end{align*}
Bis jetzt sind keine weiteren FERMATschen Primzahlen gefunden worden [4].

Sollte es so sein, dass es genau diese fünf FERMATschen Primzahlen gibt, so
folgt daraus unmittelbar, dass nur 31 (= $2^5 - 1$) regelmäßige Polygone mit
ungerader Eckenzahl mit Zirkel und Lineal konstruierbar sind. Sollte es
weitere FERMATsche Primzahlen, aber insgesamt nur endlich viele, geben, so
bliebe die Anzahl mit Zirkel und Lineal konstruierbarer regelmäßiger $n$-Ecke
mit ungeradem $n$ auch endlich. Diese Anzahl ist $2^m - 1$, dabei sei $m$ die
Anzahl der FERMATschen Primzahlen.

Nun war klar, dass das reguläre 17-Eck, aber auch das reguläre 257-Eck und das
reguläre 65537-Eck mit Zirkel und Lineal konstruierbar sind (siehe unten).
Somit waren die bekannten Konstruktionen theoretisch untermauert und es konnte
über die praktische Umsetzung der Konstruierbarkeit weiterer regulärer
$n$-Ecke nachgedacht werden.

Besonders bemerkenswert ist die Verknüpfung verschiedener Teilgebiete der
Mathematik durch GAUSS: Das Problem war geometrischer Natur, die Lösung des
Problems wurde über den Weg der Analysis in der Algebra gefunden. Rund 200
Jahre später noch einmal ein noch genialerer mathematischer „Coup“: Der Beweis
des Großen FERMATschen Satzes durch WILES 1995 [5]. Auch hierbei wurden mit
noch weitaus größerem Aufwand Erkenntnisse verschiedener mathematischer
Gebiete verbunden, um zum gewünschten Beweis zu kommen.

Die zentrale Rolle des Bestimmungswinkels bzw. seines Kosinus zur Konstruktion
des zugehörigen regulären Polygons ist in der folgenden Abbildung am Beispiel
des regulären Achtecks zu sehen: 
\begin{itemize}
\item Konstruiere Einheitskreis.
\item Konstruiere Strecke der Länge $\cos\br{\alpha=\frac{2\pi}{n}}$.
\item Konstruiere Parallele zur $y$-Achse.
\item Konstruiere Bestimmungsdreieck des regulären $n$-Ecks.
\item Konstruiere das Polygon.
\end{itemize}

\HGG{Animation der Konstruktion des regelmäßigen Achtecks einfügen}

Einige Beispiele dafür:
\begin{align*}  
  &\cos \left(\frac{2\pi}{3}\right)=\frac{1}{2},\\
  &\cos \left(\frac{2\pi}{5}\right)=\frac{1}{2}\left(-1+\sqrt{5}\right).
\end{align*}
Eine Darstellung von $\cos \left(\frac{2\pi}{7}\right)$ in ähnlicher Form als
endlich langer Term, bestehend aus reellen Zahlen, evtl. verknüpft durch die
vier Grundrechenarten und Quadratwurzeln sowie Verschachtelungen davon, ist
nicht möglich (siehe oben). 
\begin{align*}
  \cos \left(\frac{2\pi}{17}\right)&= \frac{1}{16}
  \Bigg(-1+\sqrt{17}+\sqrt{{2\left(17-\sqrt{17}\right)}}+\\
  &\qquad 2\sqrt{17+3\sqrt{17}-\sqrt{2\left(17-\sqrt{17}\right)}
    -2\sqrt{2\left(17+\sqrt{17}\right)}}\Bigg) 
\end{align*}
Ein mit \textsc{Mathematica} Version 5.2 erstellter symbolischer Ausdruck von
$\cos(2\pi/257)$ (pdf-Datei \texttt{cos2pi\_257\_symbolisch.pdf}, etwa 640
KByte) sei nach Auskunft des Softwareherstellers durch einen Bug fehlerhaft.
Eine Umwandlung von $\cos\left(\frac{2\pi}{257}\right)$ in einen symbolischen
Ausdruck mit verschachtelten Quadratwurzeln erfolgt in neueren Versionen nicht
mehr. Es wird nur die Äquivalenz
\begin{gather*}
  \cos\left(\frac{2\pi}{257}\right)=
  -\frac{1}{2}\left(\left(-1\right)^{\frac{255}{257}}\cdot
  \left(1+\left(-1\right)^{\frac{4}{257}}\right)\right)
\end{gather*}
ausgegeben.
\end{document}

Es gibt auch einen interessanten Bezug zur Graphentheorie. Um ein reguläres
Polygon zu konstruieren, reicht es, wie oben beschrieben, wenn man den
Bestimmungswinkel des regulären n-Ecks kennt. Dieser ergibt sich, wenn man
neben einem frei wählbaren Eckpunkt, sozusagen einen trivialen Eckpunkt, einen
weiteren benachbarten Eckpunkt kennt. Somit sind dann alle Eckpunkte
konstruierbar. Das bedeutet, es sind, wenn die Eckenazahl n eine FERMATsche
Primzahl ist, 5 - 1 = 4 = 2^2, 17 - 1 = 16 = 2^4, 257 - 1 = 256 = 2^8 oder
65537 - 1 = 65536 = 2^{16} "nichttriviale" Eckpunkte zu konstruieren. Wie die
nachstehende Abbildung zeigt, kann man sich das anhand vollständiger binärer
Bäume gut veranschaulichen. Wie in unten stehenden Abschnitten erklärt wird,
muss man verschachtelte quadratische Gleichungen lösen, die in unseren
betrachteten Fällen stets zwei reelle Lösungen besitzen. Jeder Knoten
entspricht einer dieser quadratischen Gleichungen, die beiden Kindknoten des
betrachteten Knotens weiteren quadratischen Gleichungen, die sich aus den
beiden Lösungen der betrachteten quadratischen Gleichung ergeben. Die
"nichttrivialen" Eckpunkte sind dann die 2^{\left({2^k} \right)} Blätter des
Baumes der Höhe 2^k (k = 1; 2; ...; 5).

Baum

Reguläres 17-Eck:
Die praktische Umsetzung der Konstruktion des regulären 17-Ecks und des
regulären 257-Ecks wurde von GAUSS nicht erbracht, sondern für das reguläre
17-Eck erstmalig 1825 von ERCHINGER [6]. Eine illustrierte
Konstruktionsbeschreibung des regulären 17-Ecks als pdf-Datei findet sich
hier.


Reguläres 257-Eck:
Die erste Konstruktion des regulären 257-Ecks beschrieb 1819, gedruckt 1822,
PAUCKER (1787-1855) [7, 8], später 1832 [9] RICHELOT (1808-1875). Für die
Konstruktion des regulären 257-Ecks folgten später weitere Arbeiten (deTEMPLE
1991 [10], GOTTLIEB 1999 [11]), die u. a. das Ziel hatten, die Anzahl der zur
Konstruktion nötigen Kreise und Geraden zu reduzieren. PAUCKER zitiert aus
einem Brief von GAUSS an ihn3 von 1820, worin GAUSS auf eigene Überlegungen
und Rechnungen zum regulären 257-Eck im Jahre 1796 hinweist. PAUCKER
vergleicht seine Ausdrücke mit denen von GAUSS und bemerkt strukturelle
Übereinstimmung in den Gleichungen, nur die Bezeichnungen waren
unterschiedlich. Das sollte bedeuten, dass bereits GAUSS einen gleichen oder
vergleichbaten Ansatz zur Herleitung verwendet hatte. GAUSS verweist explizit
auf die Verwendung der Primitivwurzel 3.([8], S. 217-219). PAUCKER benutzt als
erster auch die Schreibweise mit den indizierten Buchstaben A, B, ..., G, wie
sie auch hier in der Herleitung verwendet wird.

Zur analytischen Herleitung siehe unten.

Eine Vorstellung von der Konstruktion des regulären 257-Ecks4 im
Internetzeitalter liefert dieses Video (mp4-Format - etwa 40 MByte, avi-Format
- etwa 400 MByte). Mit Hilfe des modernen interaktiven Geometrieprogramms
"Geogebra" ( Downloadmöglichkeit) wurde die Konstruktion vom Autor auf der
Grundlage von [12] und damit von [10] durchgeführt. Zur Lösung der vielfach zu
lösenden quadratischen Gleichungen wurde ein auch für die Schulmathematik
interessanter Ansatz verwendet: Die Kreise von CARLYLE (1795-1881) [13]. Das
Prinzip ist aus der Abbildung zu erkennen.

Kreis von Carlyle

Mit einigen Hilfsschritten benötigt man für die Konstruktion der Seitenlänge
rund 380 Schritte, für das komplette 257-Eck etwa 1170 Schritte. Die
Konstruktion der Mittelsenkrechten einer Strecke wird dabei häufig
benötigt. Diese ist als "Werkzeug", gewissermaßen als Macro, vordefiniert, und
lässt sich sehr bequem in "Geogebra" nutzen. Bei ausschließlicher Verwendung
von Kreisen und Geraden ist die Anzahl der Konstruktionsschritte deutlich
größer. "Geogebra" arbeitet intern mit gerundeten Dezimalzahlen, also nicht
mit symbolischen Ausdrücken. Die Anzahl der Dezimalstellen kann man wählen. Im
Video sieht man bei einen Blick in das Algebrafenster mit welch hoher
numerischen Stabilität (vom Autor 12 Nachkommastellen gewählt) das Programm
für jede der 257 gleich langen Seiten gleiche Werte (0,02444758303)
ausrechnet.

Zwei kleine Hinweise dazu:

- Unter Verwendung der minimalen Schrittzeit von 1s bei der Darstellung einer Konstruktion in Geogebra dauert die Konstruktion auch etwa 1170 s, also etwa 20 min. Der mögliche Schnelldurchlauf des Videos gestattet eine deutliche Verringerung dieser Zeit.
- Dem Betrachter wird durch das Video auch die Auswahl einer geeigneten Vergrößerung abgenommen, um die während der Konstruktion relevanten Objekte gut sichtbar werden zu lassen.

Analytische Herleitung der Konstruktion des regulären 257-Ecks

Die moderne Begründung der Konstruierbarkeit des regulären 257-Ecks mit Zirkel
und Lineal erfolgt durch Folgerungen aus der GALOIS-Theorie. Der geneigte
Leser sei auf einschlägige Arbeiten verwiesen, u. a. in Kurzform [14],
ausführlicher BISHOP [15] mit Anmerkungen zur Konstruktion des regulären
257-Ecks.


Die Konstruktion soll mit einfacheren Mittel begründet und nachvollziehbar
hergeleitet werden. Wesentliche Stütze ist die Arbeit von deTEMPLE [10].


Der Mittelpunkt des 257-Ecks liegt im Ursprung eines komplexen
Koordinatensystems, d. h. die reellen Zahlen finden sich auf der
Abszissenachse und die Vielfachen von \mathrm{i} mit \mathrm{i}^{2}=-1 auf der
Ordinatenachse.

Alle Eckpunkte des 257-Ecks liegen auf einen Kreis. Zur Vereinfachung wählt
man den Einheitskreis mit dem Radius r = 1. Das 257-Eck wird so gedreht, dass
ein Eckpunkt der Punkt (0; 1) ist. Die Eckpunkte des 257-Ecks erfüllen die
komplexe Gleichung z^{257} = 1.

Die Gleichung z^{257} =1 soll im Bereich der komplexen Zahlen gelöst werden,
also z^{257} - 1 = 0. Der Winkel \alpha mit dem Scheitelpunkt im Ursprung des
Koordinatensystems zwischen Abszissenachse und Strahl durch den Punkt P_{\:1},
beträgt in Bogenmaß \alpha= \frac{2\pi}{257}. Also ist z = \cos\: \alpha +
\textrm{i} \: \sin \:\alpha.

So ist nach der EULERschen Identität z =
\textrm{e}^{\frac{2\pi\:\textrm{i}}{257}}. Es ist überhaupt nicht schwierig,
wenn man es mit komplexen Zahlen zu tun hat. Man kann über den Realteil von z,
also \cos\: \alpha, der auf der Abszissenachse zu finden ist, den Schnittpunkt
mit dem Einheitskreis finden. Dieser Punkt ist dann einer der Eckpunkte des
regulären 257-Ecks.

Im Bereich der komplexen Zahlen hat die Gleichung z^{257} = 1 nach dem
GAUSSschen Fundamentalsatz der Algebra 257 Lösungen. Wie man sofort sieht, ist
z = 1 eine (triviale) reelle Lösung. Die anderen 256 Lösungen ergeben sich
durch Zerlegen in Linearfaktoren mittels Polynomdivision:

\left(z - 1\right)^{257} /\left(z - 1\right) = z^{256} + z^{255} + \ldots +
z^3 + z^2 + z + 1 = 0.

Bei Division einer ganzen Zahl durch 257 können sich 257 verschiedene Reste (0, 1, ..., 256) ergeben. Man fasst alle ganzen Zahlen, die bei Division durch eine feste natürliche Zahl p, hier 257, denselben Rest lassen, zu einer Menge zusammen. Es gibt also in unserem Fall 257 paarweise verschiedene solcher Mengen. Man sagt dann, dass zwei Elemente einer Menge kongruent modulo der ganzen Zahl p sind. Die Kongruenz modulo einer ganzen Zahl ist eine Äquivalenzrelation, daher kann man die Mengen gleichen Restes als Klassen bezeichnen, kurz Restklassen. Die Menge der Restklassen modulo 257 bildet bezüglich der Multiplikation eine besondere algebraische Struktur, eine endliche Gruppe, genannt Restklassengruppe Z_{257}. Unter bestimmten Voraussetzungen kann man die Elemente einer Gruppe mit einem erzeugenden Element vollständig errechnen. Diese sind hier gegeben. Die Restklassengruppe Z_{257} kann zyklisch erzeugt werden. Es sei a Element von Z_{257}. Es gibt eine natürliche Zahl g für die gilt a = a^{g^0}, a^{g^1}, a^{g^2}, a^{g^3}, \ldots, a^{g^{258}} = a^1 oder kurz: Jede Restklasse modulo 257 kann als Potenz von a erzeugt werden, wobei der Exponent modulo 257 ist. Geometrisch ausgedrückt heißt das, wenn man neben dem trivialen Punkt (0;1) einen weiteren Punkt des 257-Ecks kennt, kann man aus dessen Lage alle restlichen 255 Punkte finden. Das wird bei der Konstruktion des 257-Ecks ausgenutzt. Eine solche Zahl g heißt Primitivwurzel modulo 257. Für alle (bekannten) FERMATschen Primzahlen außer 3 ist 3 eine Primitivwurzel. Also ist z^1 = a, z^2 = a^{3 \: mod \: 257}= a^3, z^3 = a^{9 \: mod \: 257} =a^9, z^4 = a^{27 \: mod \: 257} = a^{27}, z^5 = a^{81 \: mod \: 257} = a^{81}, z^6 = a^{243 \: mod \:257} = a^{243}, z^7= a^{729 \: mod\: 257} = a^{215}, z^8= a^{2187 \: mod \:257} = a^{165}, \ldots,
z^{257} =
a^{46336150792381577588313262263220434371406283602843045997201608143345357543255478647000589718036536507270555180182966478507
  \: mod \: 257} = a^{86}.


Bei endlichen Mengen versteht man unter der Mächtigkeit einer Menge M die
Anzahl der Elemente von M. Aus der Rechnung mit Restklassen lässt sich
offensichtlich problemlos der folgende Satz ableiten:

Es sei i die Mächtigkeit der Menge M_n der Restklassen mod n, j die
Mächtigkeit der Menge M_{2n} der Restklassen mod 2i, so ist j = 2 \cdot i und
M_n \subset M_{2n} (**)

Zur Veranschaulichung: Ein Zifferblatt einer Analogunhr ist häufig in 12
gleiche große Winkel eingeteilt. Wir geben nicht die Gesamtzahl der Stunden
seit irgendeinem Starttermin an, sondern die Restklasse mod 12. Diese ist auf
dem Zifferblatt ablesbar, z. B. als Punkt oder Strecke auf dem Schenkel des
entsprechenden Winkels. Nun soll die Angabe verfeinert werden: Jede Stunde
soll in Halbstunden eingeteilt werden. Es kommen also zusätzlich zu den
bestehenden 12 Restklassen noch 12 neue Restklassen hinzu. Die Anzahl der
Restklassen verdoppelt sich. Die ehemaligen 12 Restklassen gibt es auch bei
der neuen Einteilung noch.


Wie findet man eine nichttriviale Lösung der Gleichung z^{256} - 1 = 0?
Ein möglicher Ansatz ist die Zerlegung der Menge der 256 Lösungen in zwei
gleichmächtige Mengen M_0 und M_1. Es sollen daraus zwei reelle Zahlen A_0 und
A_1 konstruiert werden. Aus ihnen kann mithilfe des VIETAschen Wurzelsatzes
(VIETA 1540-1603) eine quadratische Gleichung bestimmt werden.

Kriterium der Auswahl ist der Index der z_i (i = 1, 2, ..., 257). Man
verwendet die Restklassen der Indices modulo 2, also 0 oder 1. (Die Restklasse
kommt in den Indices der Mengen M_0 und M_1 zum Ausdruck). So nehmen wir alle
128 Lösungen z_i = z^{\:i \:mod \: 257} mit geraden Indices in die Menge M_0,
die restlichen 128 Lösungen in die andere Menge M_1. Es sei A_0= z^3 + z^{27}
+ ... + z^{238} + z^{86} und A_1 = z^1 + z^9 + ... + z^{165} + z^{200}.


Wie man sofort sieht, ist A_0 + A_1 = -1. Das Produkt A_0 \cdot A_1 ergibt
nach Ausmultiplizieren eine Summe von 128 \cdot 128 = 16384 Summanden.


A_0 \cdot A_1 ist somit z^{1 + 3} + z^{1 + 27} + ... + z^{1 + 238} + z^{1 +
  86} + z^{9 + 3} + z^{9 + 27} + ... + z^{9 + 238} + z^{9 + 86} + ... + z^{165
  + 3} + z^{165 + 27} + ... + z^{165 + 238} + z^{165 + 65} + z^{200 + 3} +
z^{200 + 27} + ... + z^{200 + 238} + z^{200 + 86}. Wenn man die Kongruenz
modulo 257 bei den Exponenten berücksichtigt, so sieht diese Summe so aus:
z^{4} + z^{11} + ... + z^{181} + z^{29}. Zur Illustration ist in der Tabelle 1
ist diese Multiplikation vollständig dargestellt. (In der Kopfspalte sind die
Elemente der Menge M_0, in der Kopfzeile die Elemente der Menge M_1).

Mit einfachen Taschenrechnern kann man, wenn überhaupt, ohne
Genauigkeitsverlust keine natürlichen Zahlen wie 3^{256} mit 123 Ziffern
verarbeiten. Dafür gibt es Langarithmetikprogramme, z. B. in der
Mathematik-Software "Mathematik" Alpha 2016 [16] (und vielen ihrer
Vorgängerversionen). (Diese Software ist gemäß den Nutzungsbedingungen für
viele frei nutzbar:Downloadmöglichkeit.)


In Tabelle 2 ist aufgeführt, wie oft die jeweiligen Restklassen modulo 257 in
Tabelle 1 enthalten sind. Die Kopfspalte gibt den "Zehner" und die Kopfzeile
den "Einer" an, somit sind alle 257 Restklassen in der Tabelle 2. Man erkennt,
dass jede Potenz jeweils 64-mal vorkommt. So ergibt sich A_{0} \cdot A_{1} =
-64.

Die Gleichungen A_0 + A_1 = -1 und A_0 \cdot A_1 = -64 führen nach dem
VIETAschen Wurzelsatz zu der quadratischen Gleichung s^2 + s - 64 = 0 mit
A_0=\frac{-1+\sqrt{257}}{2} und A_1=\frac{-1-\sqrt{257}}{2}. A_0 und A_1 sind
beides reelle Zahlen. Man findet sie auf der Abzissenachse des komplexen
Koordinatensystems (siehe Schritt 36 der Konstruktion des regulären 257-Ecks).

Als nächstes zerlegt man die Menge der Lösungen mit geraden Indices M_{0} weiter in zwei gleichmächtige Mengen M_{0; 0} und M_{0; 2}. In der Menge M_{0; 0} sind jetzt alle Lösungen mit durch 4 teilbaren Indices und in der Menge M_{0; 2} alle Lösungen mit geradem, aber nicht durch 4 teilbaren Indices. Man wählt den Ansatz
B_0 = z^1 + z^{81} + z^{136} + ... + z^{240} + z^{165} und B_2 = z^9 + z^{215}
+ z^{196} + ... + z^{104} + z^{200}. Es ist B_0 + B_2 = A_0 und B_0 \cdot B_2
= -16. In den Tabellen 3 und 4 ist diese Multiplikation vollständig
dargestellt.


Daraus erhält man die quadratische Gleichung s^{2}- A_{0} \cdot s - 16 = 0,
deren Lösungen B_0 = \frac{A_0}{2}+ \sqrt{\left({\frac{A_0}{2}}\right)^2+16}
und B_2 = \frac{A_0}{2}- \sqrt{\left({\frac{A_0}{2}}\right)^2+16} sind.

Die restlichen Lösungen aus der Menge M_0 sind diejenigen mit ungeraden
Indices. Sie werden in zwei gleichmächtige Mengen M_{0; 1} und M_{0; 3}
aufgeteilt. In M_{0; 1} sind jetzt alle Lösungen, deren Indices bei Division
durch 4 den Rest 1 lassen, und in M_{0; 3} alle anderen noch nicht in M_{0;
  0}, M_{0; 1} und M_{0; 2} enthaltenen Lösungen, also alle deren Indices bei
Division durch 4 den Rest 3 lassen. Wir wählen B_1 = z^3 + z^{243} + z^{151} +
... + z^{206} + z^{238} und B_3 = z^{273} + z^{131} + z^{74} + ... + z^{55} +
z^{86}. Es ist B_1 + B_3 = A_1 und B_1 \cdot B_3 = -16. Man erhält daraus die
quadratische Gleichung s^2 - A_1 \cdot s - 16 = 0, deren Lösungen B_1 =
\frac{A_1}{2}+ \sqrt{\left({\frac{A_1}{2}}\right)^2+16} und B_3 =
\frac{A_1}{2}- \sqrt{\left({\frac{A_1}{2}}\right)^2+16} sind.

Zur Illustration wird auf die Tabellen 5 und 6 verwiesen.

Die Einteilung der Indices in die Mengen M_0 bis M_{0; 3} erfolgte nach ihren
Restklassen mod 2 und dann mod 4. (Es entstanden die Werte A_i\: (i = 0;
1),\:\: B_j\: (j = 0; 1; 2; 3).


Dieses Prinzip soll weiter verfolgt werden. Es werden im Folgenden mit
geeigneten Restklassen mod 8, mod 16, mod 32, mod 64 und mod 128
ausgewählt. Dann entstehen die Werte C_k\: (k = 0; 1; ...; 7), \:\: D_m\: (m =
0; 1; ...; 15),\:\: E_n\: (n = 0; 1, ...; 31),:\: F_o\: (o = 0; 1; ...; 63)
und G_p\: (p = 0; 1; ...; 127). (Die Bezeichnung der Restklassen mit diesen
Buchstaben A, B, ..., G mit den jeweiligen Indices erfolgt in gleicher Weise
wie die Bezeichnung von Punkten, die während der Konstruktion des 257-Ecks
entstehen.)

Bei jeder Einteilung der Indices in Teilmengen wird immer die gleiche Anzahl
in Klassen aufgeteilt. Aus diesem Grund gilt: \sum_{i=0}^{1}A_i =
\sum_{j=0}^{3}B_j =\sum_{k=0}^{7}C_k =\sum_{m=0}^{15}D_m =\sum_{n=0}^{31}E_n
=\sum_{o=0}^{63}F_o =\sum_{p=0}^{127}G_p =1 Wie bereits erwähnt, bilden die
Restklassen mod p, also auch mod 257, eine multiplikative Gruppe. Bei der
Multiplikation von Potenzen mit gleicher Basis wird die Basis beibehalten und
die Exponenten addiert. Es reicht also aus, wenn man die Addition der
Exponenten betrachtet. Solch eine Additionstafel ist in Tabelle 7 vollständig
dargestellt. (In der Kopfzeile bzw. -spalte sind die Elemente der Menge M_{0}
für den Fall der Restklassen 0 und 4 mod 8 angegeben.)


Nach (**) gilt C_0 + C_4 = B_0. Um mit dem Wurzelsatz von VIETA eine
quadratische Gleichung erhalten zu können, benötigt man noch einen
äquivalenten Term für C_0 \cdot C_4. Dieser soll als Linearkombination aus den
bekannten B_j\: (j = 0; 1; 2; 3) und A_i\: (i = 0; 1) gebildet werden. So ist
also

C_0 \cdot C_4 = n_{B_0} \cdot B_0 + n_{B_1} \cdot B_1 + n_{B_2} \cdot B_2 +
n_{B_3} \cdot B_3 + n_{A_0} \cdot A_0 + n_{A_1} \cdot A_1 + n_0 ,

wobei die Koeffizienten n_q\in\mathbb{Z} sein sollen. Ganze Zahlen sind mit
Zirkel und Lineal sehr leicht zu konstruieren. Man erhält somit eine
inhomogene lineare diophantische Gleichung. Zu deren Lösung gibt es
Algorithmen. Hier wird modulo 257 gerechnet, was deren Anwendung etwas
erschwert. Zum Glück ist es aber im vorliegenden Fall leicht die Lösung der
vorliegenden diophantischen Gleichung auch ohne großen Rechenaufwand zu
finden.

In Tabelle 8 wird angegeben, wie oft die einzelnen Restklassen auftreten. Man
erkennt, es gibt genau drei Fälle: 2, 4 und 5. Subtrahiert man von jeder
Anzahl 2, so erhält man 0, 2 und 3. Für Restklasse 1 mod 2 (rot markiert)
ergibt sich jeweils als Anzahl 3, und für Restklasse 2 mod 4 (blau markiert)
jeweils die Anzahl 2. Daraus ergibt sich:

C_0 \cdot C_4 = 3 \cdot \color {red} {A_1} + 2 \cdot \color {blue} {B_2} - 2. Analog erhält man:
C_1 \cdot C_5 = 3 \cdot A_0 + 2 \cdot B_3 - 2,
C_2 \cdot C_6 = 3 \cdot A_1 + 2 \cdot B_0 - 2,
C_3 \cdot C_7 = 3 \cdot A_0 + 2 \cdot B_1 - 2
. Es gilt C_0 + C_4 = B_0. Das erklärt sich sofort aus (**). So ergeben sich
auch

C_1 + C_5 = B_1, C_2 + C_6 = B_2 und C_3 + C_7 = B_3.
Eine der nach dem Satz von VIETA zu bildenden 4 quadratischen Gleichungen
lautet dann z.B. s^2 - B_0 \cdot s + 3 \cdot A_1 + 2 \cdot B_2 - 2 = 0.


Völlig analog werden als nächstens geeignete Restklassen mod 16 ausgewählt und
Linearkombinationen gebildet. Wir erhalten:

D_0 \cdot D_{ 8} = A_0 + C_0 + C_2 + 2 \cdot C_5,
D_1 \cdot D_{ 9} = A_1 + C_1 + C_3 + 2 \cdot C_6,
D_2 \cdot D_{10} = A_0 + C_2 + C_4 + 2 \cdot C_7,
D_3 \cdot D_{11} = A_1 + C_3 + C_5 + 2 \cdot C_0,
D_4 \cdot D_{12} = A_0 + C_4 + C_6 + 2 \cdot C_1,
D_5 \cdot D_{13} = A_1 + C_5 + C_7 + 2 \cdot C_2,
D_6 \cdot D_{14} = A_0 + C_6 + C_0 + 2 \cdot C_3,
D_7 \cdot D_{15} = A_1 + C_7 + C_1 + 2 \cdot C_4.
Es gilt: D_0 + D_8 = C_0, D_1 + D_9 = C_1 usw.
Es werden 8 quadratische Gleichungen gebildet, u. a. s^2 - C_0 \cdot s + A_0 +
C_0 + C_2 + 2 \cdot C_5 = 0.


Jetzt wird das Verfahren mit Restklassen mod 32 fortgesetzt. Man erhält:
E_{ 0} \cdot E_{16} = D_{ 0} + D_{ 1} + D_{ 2} + D_{ 5},
E_{ 1} \cdot E_{17} = D_{ 1} + D_{ 2} + D_{ 3} + D_{ 6},
E_{ 7} \cdot E_{23} = D_{ 7} + D_{ 8} + D_{ 9} + D_{12},
E_{ 8} \cdot E_{24} = D_{ 8} + D_{ 9} + D_{10} + D_{13},
E_{ 9} \cdot E_{25} = D_{ 9} + D_{10} + D_{11} + D_{14},
E_{31} \cdot E_{15} = D_{15} + D_{ 0} + D_{ 1} + D_{ 4}.
Es ist E_{ 0} + E_{16} = D_{ 0} usw. Diesmal werden 6 quadratische Gleichungen
gebildet, u. a. s^2 - D_0 \cdot s + D_{ 0} + D_{ 1} + D_{ 2} + D_{ 5}.

Aus den anderen Restklassen mod 32 können weitere 10 Gleichungen gebildet
werden. Sie sind aber für den Fortgang der Herleitung entbehrlich. Sie werden
nur zum Zwecke der Vollständigkeit hier aufgeführt: {\scriptsize E_{ 2} \cdot
  E_{18} = D_{ 2} + D_{ 3} + D_{ 4} + D_{ 7},\: E_{ 3} \cdot E_{19} = D_{ 3} +
  D_{ 4} + D_{ 5} + D_{ 8},\: E_{ 4} \cdot E_{20} = D_{ 4} + D_{ 5} + D_{ 6} +
  D_{ 9},\: E_{ 5} \cdot E_{21} = D_{ 5} + D_{ 6} + D_{ 7} + D_{10},\: E_{ 6}
  \cdot E_{22} = D_{ 6} + D_{ 7} + D_{ 8} + D_{11}}

{\scriptsize E_{10} \cdot E_{26} = D_{10} + D_{11} + D_{12} + D_{15},\: E_{11}
  \cdot E_{27} = D_{11} + D_{12} + D_{13} + D_{ 0},\: E_{12} \cdot E_{28} =
  D_{12} + D_{13} + D_{14} + D_{ 1},\: E_{13} \cdot E_{29} = D_{13} + D_{14} +
  D_{15} + D_{ 2},\: E_{14} \cdot E_{30} = D_{14} + D_{15} + D_{ 0} + D_{ 3}}.


Als nächstens betrachten wir Restklassen mod 64. Aus der Additionstafel (siehe
Tabelle 9) ergibt sich zunächst:


F_0 \cdot F_{32} = \color {red} {F_{33}} + \color {green} {F_{55}} + \color
{blue} {F_{23}} + \color {gray} {F_1}

Offensichtlich ist x mod 32 = (x + 32) mod 32. Daraus ergibt sich:F_{ 0} \cdot
F_{32} = E_{ 1} + E_{23}.

Analog: F_{24} \cdot F_{56} = E_{15} + E_{25}.

Es ist F_{ 0} + F_{32} = E_{ 0} und F_{24} + F_{56} = E_{24}.
Hier sind zwei quadratische Gleichungen zu bilden, um F_{24}, F_{56}, F_{ 0}
und F_{32} zu finden, also

s^2 - E_{ 0} \cdot s + E_{ 1} + E_{23} = 0, s^2 -
E_{24} \cdot s + E_{15} + F_{25} = 0

Nun noch mod 128: G_{ 0} \cdot G_{64} = F_{56}, G_{ 0} + G_{64} = F_{ 0}.
Daraus erhält man mit Hilfe von s^2 - F_{ 0} \cdot s + F_{56} = 0
u. a. G_{64}.


Es wurde jener Wert für z gefunden, dessen Exponent beim Potenzieren von 3 mod
257 den Rest 64 ergibt. Nach Tabelle 1 ist ein 32 ein solcher Wert. Das heißt
z=\textrm{e}^{\left(\frac{2\pi \textrm{i}}{257}\right)^{32}}=
\textrm{e}^{\left(\frac{2\pi \textrm{i} \cdot 32}{257}\right)}
=\textrm{e}^{\left(32\cdot\frac{2\pi \textrm{i}}{257}\right)} und damit der
Realteil \cos\left(32\cdot\alpha\right). Auf den ersten Blick führt aber
G_{64} \approx 1.8 zu einem Widerspruch. G_{64} gibt geometrisch gesehen die
Sehnenlänge zwischen zwei Eckpunkten P_{\:i} und P_{\:i+32} (i = 0; \ldots;
256) des 257-Ecks an. Das bedeutet: G_{64}/2 ist die halbe Sehnenlänge, ist
auch eine Lösung der Gleichung x^{256} - 1 = 0 und entspricht damit \cos
\left({16\alpha}\right). G_{64}/2 ist die Abszisse des Eckpunktes P_{17} des
regulären 257-Ecks. Mit der fortwährenden Abtragung der Sehnenlänge
\overline{P_{\:1}P_{17}} (mit dem Zirkel) erhält man alle weiteren Eckpunkte
des regulären 257-Ecks: In der beigefügten Konstruktion des regulären 257-Eck
Schritt 398 bis 911.


Reguläres 65537-Eck:

Die Umsetzung der Konstruierbarkeit des regulären 65537-Ecks hatte sich HERMES
(1846-1912) vorgenommen. Er legte die Arbeit [17] 1894 vor. Grundlage war
[18]. Obwohl in der Arbeit das 65537-Eck nicht konstruiert wird, gibt sie
einen genauen Überblick, wie man es machen könnte. Felix KLEIN schlug vor,
diese Arbeit als Dissertation anzuerkennen. Bekanntermaßen wird sie heute in
einem speziellen mit Stoff ausgeschlagenen Holzkoffer im Mathematischen
Institut der Georg-August-Universität in Göttingen aufbewahrt. Der Koffer ist
Teil der Modellsammlung des Mathematischen Instituts der Universität
Göttingen.

Der Autor bekam anlässlich der Bundesrunde der 49. Mathematik-Olympiade 2010
in Göttingen die besondere Gelegenheit geboten, die besagte Arbeit persönlich
in Augenschein nehmen und darin lesen zu dürfen. Auch durften Fotos
angefertigt werden.

Die Arbeit umfasst 221 Seiten, die in einer der Deutschen Kurrentschrift
ähnlichen Schrift geschrieben sind. Die Seiten haben z. T. unterschiedliches
Format. Einige davon mussten zusammengeklappt werden, um ein aus heutiger
Sicht unstandardisiertes Format, dass zwischen A4- und A3-Größe liegt, zu
erhalten. Einige Seiten sind aus mehreren Stücken zusammengeklebt worden
bzw. es sind Korrekturen eingearbeitet worden. Möglicherweise hätte eine
Abschrift der betreffenden Seiten einen zu hohen Zeitaufwand bedeutet, der den
Anfertigungszeitraum der Arbeit noch weiter ausgedehnt hätte. Wenn man die
Seiten betrachtet, gewinnt man einen Eindruck vom außerordentlichen Fleiß und
der Ausdauer von HERMES, der ab 1879 mehr als 10 Jahre daran arbeitete. Leider
können die angefügten Bilder nur einen unzureichenden Eindruck davon
vermitteln. In der Arbeit werden sehr systematisch Betrachtungen zu Lösungen
der entsprechenden Kreisteilungsgleichungen gemacht, die zur Konstruktion des
regulären 5-, 17-, 257- und 65537-Eck führen. Schon im benötigten Platz für
seine Betrachtungen zu diesen Polygonen wird der immens steigende Rechen- und
Schreibaufwand deutlich, um die Lösungen der Kreisteilungsgleichungen zu
finden. Dazu hat er eigene Symbole entwickelt, um die Übersichtlichkeit und
Kompaktheit der Darstellung zu erhöhen. HERMES macht in seiner Arbeit
deutlich, dass er nach Analogien, Perioden und ähnlichem in den Lösungen der
Kreisteilungsgleichung für das reguläre 65537-Eck gesucht hat5 .


Im Folgenden werden einige Bemerkungen bei der Herleitung der Konstruktion des
regulären 65537-Ecks gemacht, die auch die immense Arbeitsleistung von HERMES
noch mehr verdeutlichen.

BRAKKE hat mit einem sehr ähnlichen Ansatz, wie oben im Text beim regulären
17-Eck und 257-Eck beschrieben, verschachtelte quadratische Gleichungen mit
Computerunterstützung hergeleitet und als Mathematica-File veröffentlicht
[19]. Damit lässt sich in kurzer Zeit die Abszisse des 8192. Eckpunktes des
regulären 65537-Ecks berechnen.

Der Autor hat die Gleichungen so umgeformt und umkodiert, dass die von ihm oben benutzte Notation der Restklassen verwendet werden kann. So lässt sich das reguläre 65537-Eck, wieder ähnlich wie das reguläre 257-Eck, konstruieren.
Die Anzahl der ineinander verschachtelten quadratischen Gleichungen verdoppelt
sich mit jeder Erhöhung der "Verteilebene" A, B, C, ... ,R6. Ein Eckpunkt ist
wieder mit dem Punkt (1;0) frei gewählt.

Da aber die Herleitung der Abszisse eines weiteren Eckpunktes, hier des
8192. Eckpunktes7 , letztlich ausreicht um daraus das gesamte reguläre
65537-Eck zu konstruieren, muss man nicht alle quadratischen Gleichungen
lösen. Ab der Verteilebene K benötigt man immer weniger Gleichungen (siehe
nachstehende Tabelle 10). Die Gesamtanzahl der benutzten quadratischen
Gleichungen beträgt 1141. (Zum Vergleich: Beim regulären 257-Eck waren es
"nur" 24.) Nicht immer benötigt man beide Lösungen für die weitere
Rechnung. Die nicht weiter verwendeten Lösungen werden in der Tabelle
durchgekreuzt. Die komplette Darstellung aller Gleichungen würde sehr viel
Platz beanspruchen. Aus diesem Grund können die Gleichungen in einer pdf-Datei
( Verschachelte quadratische Gleichungen zur Berechnung von cos(2pi/65537))
betrachtet werden. In Tabelle 10 sind einige Beispiele angegeben.

Ausgangspunkt der Überlegungen ist wieder der beim regulären 17- und 257-Eck
erfolgreich genutzte Ansatz die Restklassen (außer der Restklasse 0), jetzt
mod 65537, in zwei gleichmächtige Mengen A0 und A1 aufzuteilen. Der weitere
Weg ist ebenfalls oben ausführlich beschrieben. Beim regulären 257-Eck wurde
die Bildung der Linearkombinationen aus den Restklassen anschaulich mit den
Multiplikationstafeln beschrieben. Ein viel effektiverer Weg, vor allem mit
Computerhilfe, für die Ermittlung jeder der Linearkombinationen ist, lineare
Gleichungssysteme aufzustellen und dann zu lösen.

Deutlich wird, dass man die quadratischen Gleichungen in einer bestimmten
Verteilebene nur mit den Lösungen der quadratischen Gleichungen voriger
Verteilebenen aufstellen kann. In den Linearkombinationen sind Lösungen
quadratischer Gleichungen von bis zu 7 vorhergehender Verteilebenen
enthalten. Für die Aufstellung der quadratischen Gleichungen kommt der Satz
von VIETA zur Anwendung.

Verteilebene 	mod n 	Anzahl der quadratischen Gleichungen 	davon
verwendet für weitere Rechnungen 	Beispiel

A 	2 	1 	1 	A0 + A1 = -1
A0 ∙ A1 = -16384

quadratische Gleichung: s2 + 1 ∙ s - 16384 = 0
B 	4 	2 	2 	B0 + B2 = A0
B0 ∙ B2 = - 4096
allgemein: Bi ∙ Bi+2= -4096 (i= 0, 1)

quadratische Gleichung: s2 - A0 ∙ s - 4096 = 0
C 	8 	4 	4 	C0 + C4 = B0
C0 ∙ C4 = - 16 ∙ A0 - 32 ∙ B0 - 1040
allgemein: Ci ∙ Ci+4 = -16 ∙ A(i div 2) mod 2 - 32 ∙ B(i mod 2) + (i div 2) - 1040
(i= 0, 1, 2, 3)

quadratische Gleichung: s2 - B0 ∙ s -16 ∙ A0 - 32 ∙ B0 - 1040 = 0
D 	16 	8 	8 	D0 + D8 = C0
D0 ∙ D8 = + 19 ∙ A0 + 32 ∙ B3 + 28 ∙ C5 - 32 ∙ C3 + 16 ∙ C0 - 237
allgemein: Di ∙ Di+8 = 19 ∙ Ai mod 2 + 32 ∙ B(i+3) mod 4
+ 28 ∙ C(i mod 2) + (i mod 4) div 2 + 5 ∙ ((i mod 4) +1) + 4 ∙ (i div 4)) mod 8
- 32 ∙ C((i+1 ) mod 2 + ((i+1) mod 4)div 2 + 5 ∙ ((i+1) mod 4 + 1) + 4 ∙ (((i+1) mod 8 ) div 4)) mod 8
+ 16 ∙ C((i+2) mod 2 +((i+2) mod 4)div 2 + 5 ∙ ((i+2) mod 4) + 1) + 4 ∙ (((i+2) mod 8) div 4 )) mod 8
(i= 0, 1, ..., 7)

quadratische Gleichung: s2 - C0 ∙ s + 19 ∙ A0 + 32 ∙ B3 + 28 ∙ C5 - 32 ∙ C3 + 16 ∙ C0 - 237 = 0

E
	

32
	

16
	

16
	E0 + E16 = D0
E0 ∙ E16 = -9 ∙ A0 + 4 ∙ B2 + 3 ∙ C5 - 6 ∙ C3 + 3 ∙ C0 - 12 ∙ C6 + 12 ∙ D0 - 2 ∙ D1 - 4 ∙ D2 + 6 ∙ D3 - 8 ∙ D4 - 10 ∙ D5 - 10 ∙ D7 - 70
allgemein: 14 Summanden

quadratische Gleichung: s2 - D0 ∙ s - 9 ∙ A0 + 4 ∙ B2 + 3 ∙ C5 - 6 ∙ C3 + 3 ∙ C0 - 12 ∙ C6 + 12 ∙ D0 -2 ∙ D1 - 4 ∙ D2 + 6 ∙ D3 - 8 ∙ D4 - 10 ∙ D5 - 10 ∙ D7 - 70 = 0

F
	

64
	

32
	

32
	F0 + F32 = E0
F0 ∙ F32 = 2 ∙ A0 + 6 ∙ B3 + 16 ∙ C5 - C3 - 6 ∙ C0 + 6 ∙ C6 - 9 ∙ D0 - D1 + 18 ∙ D2 - 5 ∙ D3 + 3 ∙ D4 -5 ∙ D5 + 3 ∙ D6 + 6 ∙ D7 - 16 ∙ E0 -3 ∙ E1 - 5 ∙ E2 + E3 + 4 ∙ E4 + 6 ∙ E5 + 2 ∙ E7 - 10 ∙ E8 + 6 ∙ E9 + 5 ∙ E10 + 5 ∙ E11 - 4 ∙ E13 - 2 ∙ E14 + E15 - 11
allgemein: 29 Summanden

quadratische Gleichung: s2 - E0 ∙ s + 2 ∙ A0 + 6 ∙ B3 + 16 ∙ C5 - C3 - 6 ∙ C0 + 6 ∙ C6 - 9 ∙ D0 - D1 + 18 ∙ D2 - 5 ∙ D3 + 3 ∙ D4 - 5 ∙ D5 + 3 ∙ D6 + 6 ∙ D7 - 16 ∙ E0 - 3 ∙ E1 - 5 ∙ E2 + E3 + 4 ∙ E4 + 6 ∙ E5 + 2 ∙ E7 - 10 ∙ E8 + 6 ∙ E9 + 5 ∙ E10 + 5 ∙ E11 - 4 ∙ E13 - 2 ∙ E14 + E15 - 11 = 0

G
	

128
	

64
	

64
	G0 + G64 = F0
G0 ∙ G64 = 5 ∙ A0 - 5 ∙ B2 + 2 ∙ B3 - C5 - C3 - 5 ∙ C0 + C6 + 5 ∙ D0 + 3 ∙ D1 + D2 + 2 ∙ D4 + D5 - D7 - 6 ∙ E0 - E1 - E2 - 2 ∙ E3 - 4 ∙ E4 - 5 ∙ E5 - E6 + E7 + E8 - E9 + E10 + 2 ∙ E11 - 3 ∙ E13 - 5 ∙ E14 + E15 - F0 - 3 ∙ F1 - F2 + 3 ∙ F3 + 4 ∙ F4 + 4 ∙ F5 - 5 ∙ F6 + 2 ∙ F7 - F8 + 3 ∙ F9 + F10 - 2 ∙ F12 + 4 ∙ F13 + F15 - F16 + F17 - F18 + F19 - 3 ∙ F20 - F21 - 3 ∙ F22 - F23 + F24 + 6 ∙ F25 - F27 + F28 - 7 ∙ F30 + F31 - 3
allgemein: 57 Summanden

quadratische Gleichung: s2 - F0 ∙ s + 5 ∙ A0 - 5 ∙ B2 + 2 ∙ B3 - C5 - C3 - 5 ∙ C0 + C6 + 5 ∙ D0 + 3 ∙ D1 + D2 + 2 ∙ D4 + D5 - D7 - 6 ∙ E0 - E1 - E2 - 2 ∙ E3 - 4 ∙ E4 - 5 ∙ E5 - E6 + E7 + E8 - E9 + E10 + 2 ∙ E11 - 3 ∙ E13 - 5 ∙ E14 + E15 - F0 - 3 ∙ F1 - F2 + 3 ∙ F3 + 4 ∙ F4 + 4 ∙ F5 - 5 ∙ F6 + 2 ∙ F7 - F8 + 3 ∙ F9 + F10 - 2 ∙ F12 + 4 ∙ F13 + F15 - F16 + F17 - F18 + F19 - 3 ∙ F20 - F21 - 3 ∙ F22 - F23 + F24 + 6 ∙ F25 - F27 + F28 - 7 ∙ F30 + F31 - 3 = 0

H
	

256
	

128
	

128
	H0 + H128 = G0
H0 ∙ H128 = -2 ∙ A0 + B2 - B3 - C5 + C3 - C6 - 2 ∙ D1 + 4 ∙ D2 + D5 + D6 - D7 + 2 ∙ E1 - 3 ∙ E2 + 2 ∙ E4 - E6 + E7 + 2 ∙ E8 + E9 + 4 ∙ E10 - E11 + E12 + E13 + E14 - E15 + 2 ∙ F1 - F2 + F3 - 3 ∙ F4 - 2 ∙ F5 + 2 ∙ F6 - 2 ∙ F7 - 2 ∙ F8 - F9 -2 ∙ F10 - 2 ∙ F13 - F14 + F15 + F16 - 3 ∙ F18 + F19 - F20 - 2 ∙ F21 + F23 + 2 ∙ F24 - F25 + F26 - F27 - F28 - F29 + 2 ∙ F30 - F31 + 2 ∙ G0 - 3 ∙ G1 - G3 + G4 - G6 + G7 + G8 - 2 ∙ G9 + G10 - 2 ∙ G12 + G14 - G15 - G18 - 2 ∙ G19 + G20 - 2 ∙ G23 -2 ∙ G24 + 3 ∙ G29 - G30 - G31 + G34 - G36 - G37 + 2 ∙ G38 - 2 ∙ G39 - G41 - 2 ∙ G42 - 2 ∙ G44 - G45 - G46 + G48 - 4 ∙ G50 - G52 - 2 ∙ G53 - G54 - G55 + G56 + 3 ∙ G57 + G58 + G59 - 2
allgemein: 92 Summanden

quadratische Gleichung: s2 - G0 ∙ s - 2 ∙ A0 + B2 - B3 - C5 + C3 - C6 - 2 ∙ D1 + 4 ∙ D2 + D5 + D6 - D7 + 2 ∙ E1 - 3 ∙ E2 + 2 ∙ E4 - E6 + E7 + 2 ∙ E8 + E9 + 4 ∙ E10 - E11 + E12 + E13 + E14 - E15 + 2 ∙ F1 - F2 + F3 -3 ∙ F4 - 2 ∙ F5 + 2 ∙ F6 - 2 ∙ F7 - 2 ∙ F8 - F9 -2 ∙ F10 - 2 ∙ F13 - F14 + F15 + F16 - 3 ∙ F18 + F19 - F20 - 2 ∙ F21 + F23 + 2 ∙ F24 - F25 + F26 - F27 - F28 - F29 + 2 ∙ F30 - F31 + 2 ∙ G0 - 3 ∙ G1 - G3 + G4 - G6 + G7 + G8 - 2 ∙ G9 + G10 - 2 ∙ G12 + G14 - G15 - G18 - 2 ∙ G19 + G20 - 2 ∙ G23 - 2 ∙ G24 + 3 ∙ G29 - G30 - G31 + G34 - G36 - G37 + 2 ∙ G38 - 2 ∙ G39 - G41 - 2 ∙ G42 - 2 ∙ G44 - G45 - G46 + G48 - 4 ∙ G50 - G52 - 2 ∙ G53 - G54 - G55 + G56 + 3 ∙ G57 + G58 + G59 -2 = 0

J
	

512
	

256
	

256
	J0 + J256 = H0
J0 ∙ J256 = B2 + B3 - C5 - C3 + C0 - D2 + D3 - D4 + D5 - E3 - 2 ∙ E5 + E7 + E9 - E10 - E12- E13 + F3 - F7 + F8 - F9 - F19 + F20 - 2 ∙ F21 - F26 + F27 - F28 - F29 + G0 - G3 + G7 - G8 + G14 + G18 - G20 - G27 + 2 ∙ G28 - G39 - G41 + G45 + G47 - G51 - 2 ∙ G53 + 2 ∙ G57 + G59 - G61 - H0 + 2 ∙ H3 + H4 - H18 + H19 + H29 + H30 + H33 + H37 + H39 + H41 + H43 + H44 - H45 - H47 + H50 + H51 + H53 + H56 - 2 ∙ H57 - H58 - H60 + H61 + H65 - H67 + H68 + H70 - H72 + H78 + H79 + H81 + 3 ∙ H82 + H88 + H89 - H91 - H103 - H105 + H106 + H109 + H112 - H115 - H117 - H122 - H125
allgemein: 89 Summanden

quadratische Gleichung: s2 - H0 ∙ s + B2 + B3 - C5 - C3 + C0 - D2 + D3 - D4 + D5 - E3 - 2 ∙ E5 + E7 + E9 - E10 - E12- E13 + F3 - F7 + F8 - F9 - F19 + F20 - 2 ∙ F21 - F26 + F27 - F28 - F29 + G0 - G3 + G7 - G8 + G14 + G18 - G20 - G27 + 2 ∙ G28 - G39 - G41 + G45 + G47 - G51 - 2 ∙ G53 + 2 ∙ G57 + G59 - G61 - H0 + 2 ∙ H3 + H4 - H18 + H19 + H29 + H30 + H33 + H37 + H39 + H41 + H43 + H44 - H45 - H47 + H50 + H51 + H53 + H56 - 2 ∙ H57 - H58 - H60 + H61 + H65 - H67 + H68 + H70 - H72 + H78 + H79 + H81 + 3 ∙ H82 + H88 + H89 - H91 - H103 - H105 + H106 + H109 + H112 - H115 - H117 - H122 - H125 = 0

K
	

1024
	

512
	

481
	K0 + K512 = J0
K0 ∙ K512 = 2 ∙ F2 + F16 - 2 ∙ G2 + G5 - G16 + G46 - H5 + H12 + H18 + H21 + H24 - 2 ∙ H66 - H80 + H92 + H122 - J12 - J18 - J21 - J24 + J40 - J46 + J48 + 2 ∙ J67 + J80 + 2 ∙ J85 + 2 ∙ J87 + J91 - J92 + J105 + J108 + J113 - J122 - J133 + 2 ∙ J134 + J174 - 2 ∙ J194 - J208 + J210 + J225 + J232
allgemein: 40 Summanden

quadratische Gleichung: s2 - K0 ∙ s + 2 ∙ F2 + F16 - 2 ∙ G2 + G5 - G16 + G46 - H5 + H12 + H18 + H21 + H24 - 2 ∙ H66 - H80 + H92 + H122 - J12 - J18 - J21 - J24 + J40 - J46 + J48 + 2 ∙ J67 + J80 + 2 ∙ J85 + 2 ∙ J87 + J91 - J92 + J105 + J108 + J113 - J122 - J133 + 2 ∙ J134 + J174 - 2 ∙ J194 - J208 + J210 + J225 + J232 = 0

L
	

2024
	

1024
	

110
	L0 + L1024 = K0
L0 ∙ L1024 = G49 + 2 ∙ H9 - H49 + H72 - 2 ∙ J9 + J18 - J72 + J154 - J177 + J206 + J221 + 2 ∙ K0 - K18 + K23 + K154 + K184 - K206 - K221 - 2 ∙ K265 + K308 - K328 + K359 - K433
allgemein: 23 Summanden

quadratische Gleichung: s2 - K0 ∙ s + G49 + 2 ∙ H9 - H49 + H72 - 2 ∙ J9 + J18 - J72 + J154 - J177 + J206 + J221 + 2 ∙ K0 - K18 + K23 + K154 + K184 - K206 - K221 - 2 ∙ K265 + K308 - K328 + K359 - K433 = 0

M
	

4096
	

2024
	

30
	M92 + \xcancel{M_{2140}}= L92
M92 ∙ \xcancel{M_{2140}} = K167 + K246 + K331 + L92 + L93 + L94 - L167 - L246 - L331 + L869 + L892
allgemein: 11 Summanden

quadratische Gleichung: s2 - L92 ∙ s + K167 + K246 + K331 + L92 + L93 + L94 - L167 - L246 - L331 + L869 + L892 = 0

N
	

8192
	

4096
	

6
	N5120 + \xcancel{N_{1024}} = M1024
N5120 ∙ \xcancel{N_{1024}} = G35 - H35 - J163 - K419 - L1955 + M1025 + M2289 + M2923 - M2979
allgemein: 9 Summanden

quadratische Gleichung: s22 - M1024 ∙ s + G35 - H35- J163 - K419 - L1955 + M1025 + M2289 + M2923 - M2979 = 0

P
	

16384
	

8192
	

2
	P13312 + \xcancel{P_{5120}} = N5120
P13312 ∙ \xcancel{P_{5120}}= G36 - H36 - J164 - K420 - L1956 - M2980 + N4188 - N5028
allgemein: 8 Summanden

quadratische Gleichung: s2 - N5120 ∙ s + G36 - H36 - J164 - K420 - L1956 - M2980 + N4188 - N5028 = 0

R
	

32768
	

16384
	

1
	R30720 + \xcancel{R_{14336}} =P14336
R30720 ∙ \xcancel{R_{14336}} = P13312

quadratische Gleichung: s2 - P14336 ∙ s + P13312 = 0

Summe
	

---
	

---
	

1141
	

---


Zur Wahrung der Vollständigkeit sei hier angeführt, dass das Prinzip der
Aufteilung der Restklassen in geeignete Mengen, wie nicht anders zu erwarten,
für die Konstruktion aller regulären Polygone mit einer Eckenzahl gleich einer
der FERMATschen Primzahlen angewandt werden kann, also auch bei der
Konstruktion des regulären 5-Ecks und 3-Ecks. Jedoch führen bekanntermaßen
einfachere Überlegungen schneller zum Ziel.


Der Autor dankt an dieser Stelle noch einmal ausdrücklich allen Beteiligten
für die Unterstützung in Göttingen und die Möglichkeit Text und Fotos hier
veröffentlichen zu dürfen!

\end{document}

Koffer 65537-Eck 	Koffer 65537-Ecks offen 	Diarium ... 	Konstruktion des 65537-Ecks
Einleitung 	Hinweis Fermatsche Zahl 	Übersicht verschiedene Tabellen 	Ausführungen zum 5-Eck
Ausführungen zum 17-Eck 	Ausführungen zum 257-Eck 	Hinweise zur Klebetechnik 	Einweis zum 65537-Eck
Berechnung der Quadratwurzel aus 65537 	Gleichungen zum 65537-Eck 	Hinweis auf Kodierung der Indices 	Indices-Tabellen
Tabellen und Symbolen 	Tabellen mit Symbolen 	Tabellen mit Symbolen, vergrößert 	Tabellen mit Zahlen
Anliegen und Resumeé der Arbeit 	letzte ausgeführte Seite 	Der Autor mit dem Koffer

Literatur und andere Quellen:
[1] Mathematisches Tagebuch 1796-1814 von Carl Friedrich Gauss. Mit einer historischen Einführung von Kurt-R. Biermann (Ostwalds Klassiker der exakten Wissenschaften, Band 256, Begründet von Wilhelm Ostwald). Hrsg. D. Goetz, E. Wächtler, H. Wußing, Akademische Verlagsgesellschaft Geest & Portig, K.-G., Leipzig 1976, S. 15f.
[2] Patzschke, Jürgen: Carl Friedrich Gauss und das Siebzehneck - Teil I, Zeitschrift "Wurzel", Hrsg. Wurzel-Verein zur Förderung der Mathematik an Schulen und Universitäten, Jena 2002, Heft 5, S.15-24
[3] http://de.wikipedia.org/wiki/Wantzel (abgerufen am 01.09.2011)
[4] http://de.wikipedia.org/wiki/Fermat-Zahl (abgerufen am 16.05.2012)
[5] http://de.wikipedia.org/wiki/Gro%C3%9Fer_Fermatscher_Satz (abgerufen am 16.05.2012)
[6] http://de.wikipedia.org/wiki/Siebzehneck (abgerufen am 01.09.2011)
[7] http://de.wikipedia.org/wiki/257-Eck (abgerufen am 03.06.2016)
[8] Paucker, Georg: Geometrische Verzeichnung des regelmässigen Siebzehn-Ecks und Zweyhundertsiebenundfunfzig-Ecks in den Kreis, In: Jahresverhandlungen der kurländischen Gesellschaft für Literatur und Kunst, Zweyter Band, Mitau 1822, S. 160-219;
[9] Richelot, Friedrich Julius: De resolutione algebraica aequationis x257 = 1, sive de divisione circuli per bisectionem anguli septies repetitam in partes 257 inter se aequales commentatio coronata. In: Journal für die reine und angewandte Mathematik. Nr. 9, 1832, S. 1-26, 146-161, 209-230, und 337-358.
[10] DeTemple, Duane W.: Carlyle Circles and the Lemoine Simplicity of Polygonal Constructions. In: The American Mathematical Monthly. No. 98, 1991, S. 97-108.
[11] Gottlieb, Christian: The Simple and Straightforward Construction of the Regular 257-gon. In: Mathematical Intelligencer. Vol. 21, No. 1, 1999, S. 31-37.
[12] http://it.wikipedia.org/wiki/257-gono (abgerufen am 01.09.2011)
[13] http://it.wikipedia.org/wiki/Cerchio_di_Carlyle (abgerufen am 01.09.2011)
[14] http://de.wikipedia.org/wiki/Konstruierbare%20Polygone (abgerufen am 20.05.2016)
[15] Bishop, Wayne: How to construct a regular polygon, Amer. Math Monthly, 85 (1978) 186-188
[16] Polster, Steffen: "Mathematik" Alpha 2016, http://mathematikalpha.de
[17] Hermes, Johann Gustav: Diarium zur Kreisteilung, Königsberg 1879, (1879 begonnen - d. A.)
[18] Hermes, Johann Gustav: Ueber die Teilung des Kreises in 65537 gleiche Teile. In: Nachrichten von der Gesellschaft der Wissenschaften zu Göttingen, Mathematisch-Physikalische Klasse. Göttingen, 1894, S. 170–186.
[19] http://facstaff.susqu.edu/brakke/constructions/65537-gon.m.txt (abgerufen am 06.06.2017) 1Historie: erstellt 2012, bearbeitet und ergänzt am 24.06.2016, 30.06.2017 und 12.06.2019
2Nichtnegative ganze Zahlen sind: 0; 1; 2; 3; ...
3GAUSS gibt in diesem Brief den 30. März als Tag seiner Entdeckung der Kreisteilung an ([8], S. 219). Ein Tag später als in der oben zitierten Aussage von GAUSS.
4anlässlich des Festvortrages im Rahmen der Feierlichen Siegerehrung der Landesrunde Sachsen der 50. Mathematik-Olympiade der Klassen 9 bis 12 am 27.02.2011 in Leipzig
5Möglicherweise kann der eine oder andere Betrachter der Bilder keine Deutsche Kurrentschrift lesen, daher sind vom Autor einige Bemerkungen in die Bilder eingefügt worden.
6Zur Vermeidung von Verwechslungen mit Ziffern wurde auf die Buchstaben I, O und Q verzichtet.
7 8192 = 65536 : 8

\end{document}
